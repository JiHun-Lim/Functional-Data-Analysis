% Options for packages loaded elsewhere
\PassOptionsToPackage{unicode}{hyperref}
\PassOptionsToPackage{hyphens}{url}
%
\documentclass[
]{article}
\usepackage{amsmath,amssymb}
\usepackage{lmodern}
\usepackage{ifxetex,ifluatex}
\ifnum 0\ifxetex 1\fi\ifluatex 1\fi=0 % if pdftex
  \usepackage[T1]{fontenc}
  \usepackage[utf8]{inputenc}
  \usepackage{textcomp} % provide euro and other symbols
\else % if luatex or xetex
  \usepackage{unicode-math}
  \defaultfontfeatures{Scale=MatchLowercase}
  \defaultfontfeatures[\rmfamily]{Ligatures=TeX,Scale=1}
\fi
% Use upquote if available, for straight quotes in verbatim environments
\IfFileExists{upquote.sty}{\usepackage{upquote}}{}
\IfFileExists{microtype.sty}{% use microtype if available
  \usepackage[]{microtype}
  \UseMicrotypeSet[protrusion]{basicmath} % disable protrusion for tt fonts
}{}
\makeatletter
\@ifundefined{KOMAClassName}{% if non-KOMA class
  \IfFileExists{parskip.sty}{%
    \usepackage{parskip}
  }{% else
    \setlength{\parindent}{0pt}
    \setlength{\parskip}{6pt plus 2pt minus 1pt}}
}{% if KOMA class
  \KOMAoptions{parskip=half}}
\makeatother
\usepackage{xcolor}
\IfFileExists{xurl.sty}{\usepackage{xurl}}{} % add URL line breaks if available
\IfFileExists{bookmark.sty}{\usepackage{bookmark}}{\usepackage{hyperref}}
\hypersetup{
  pdftitle={HW1},
  hidelinks,
  pdfcreator={LaTeX via pandoc}}
\urlstyle{same} % disable monospaced font for URLs
\usepackage[margin=1in]{geometry}
\usepackage{color}
\usepackage{fancyvrb}
\newcommand{\VerbBar}{|}
\newcommand{\VERB}{\Verb[commandchars=\\\{\}]}
\DefineVerbatimEnvironment{Highlighting}{Verbatim}{commandchars=\\\{\}}
% Add ',fontsize=\small' for more characters per line
\usepackage{framed}
\definecolor{shadecolor}{RGB}{248,248,248}
\newenvironment{Shaded}{\begin{snugshade}}{\end{snugshade}}
\newcommand{\AlertTok}[1]{\textcolor[rgb]{0.94,0.16,0.16}{#1}}
\newcommand{\AnnotationTok}[1]{\textcolor[rgb]{0.56,0.35,0.01}{\textbf{\textit{#1}}}}
\newcommand{\AttributeTok}[1]{\textcolor[rgb]{0.77,0.63,0.00}{#1}}
\newcommand{\BaseNTok}[1]{\textcolor[rgb]{0.00,0.00,0.81}{#1}}
\newcommand{\BuiltInTok}[1]{#1}
\newcommand{\CharTok}[1]{\textcolor[rgb]{0.31,0.60,0.02}{#1}}
\newcommand{\CommentTok}[1]{\textcolor[rgb]{0.56,0.35,0.01}{\textit{#1}}}
\newcommand{\CommentVarTok}[1]{\textcolor[rgb]{0.56,0.35,0.01}{\textbf{\textit{#1}}}}
\newcommand{\ConstantTok}[1]{\textcolor[rgb]{0.00,0.00,0.00}{#1}}
\newcommand{\ControlFlowTok}[1]{\textcolor[rgb]{0.13,0.29,0.53}{\textbf{#1}}}
\newcommand{\DataTypeTok}[1]{\textcolor[rgb]{0.13,0.29,0.53}{#1}}
\newcommand{\DecValTok}[1]{\textcolor[rgb]{0.00,0.00,0.81}{#1}}
\newcommand{\DocumentationTok}[1]{\textcolor[rgb]{0.56,0.35,0.01}{\textbf{\textit{#1}}}}
\newcommand{\ErrorTok}[1]{\textcolor[rgb]{0.64,0.00,0.00}{\textbf{#1}}}
\newcommand{\ExtensionTok}[1]{#1}
\newcommand{\FloatTok}[1]{\textcolor[rgb]{0.00,0.00,0.81}{#1}}
\newcommand{\FunctionTok}[1]{\textcolor[rgb]{0.00,0.00,0.00}{#1}}
\newcommand{\ImportTok}[1]{#1}
\newcommand{\InformationTok}[1]{\textcolor[rgb]{0.56,0.35,0.01}{\textbf{\textit{#1}}}}
\newcommand{\KeywordTok}[1]{\textcolor[rgb]{0.13,0.29,0.53}{\textbf{#1}}}
\newcommand{\NormalTok}[1]{#1}
\newcommand{\OperatorTok}[1]{\textcolor[rgb]{0.81,0.36,0.00}{\textbf{#1}}}
\newcommand{\OtherTok}[1]{\textcolor[rgb]{0.56,0.35,0.01}{#1}}
\newcommand{\PreprocessorTok}[1]{\textcolor[rgb]{0.56,0.35,0.01}{\textit{#1}}}
\newcommand{\RegionMarkerTok}[1]{#1}
\newcommand{\SpecialCharTok}[1]{\textcolor[rgb]{0.00,0.00,0.00}{#1}}
\newcommand{\SpecialStringTok}[1]{\textcolor[rgb]{0.31,0.60,0.02}{#1}}
\newcommand{\StringTok}[1]{\textcolor[rgb]{0.31,0.60,0.02}{#1}}
\newcommand{\VariableTok}[1]{\textcolor[rgb]{0.00,0.00,0.00}{#1}}
\newcommand{\VerbatimStringTok}[1]{\textcolor[rgb]{0.31,0.60,0.02}{#1}}
\newcommand{\WarningTok}[1]{\textcolor[rgb]{0.56,0.35,0.01}{\textbf{\textit{#1}}}}
\usepackage{graphicx}
\makeatletter
\def\maxwidth{\ifdim\Gin@nat@width>\linewidth\linewidth\else\Gin@nat@width\fi}
\def\maxheight{\ifdim\Gin@nat@height>\textheight\textheight\else\Gin@nat@height\fi}
\makeatother
% Scale images if necessary, so that they will not overflow the page
% margins by default, and it is still possible to overwrite the defaults
% using explicit options in \includegraphics[width, height, ...]{}
\setkeys{Gin}{width=\maxwidth,height=\maxheight,keepaspectratio}
% Set default figure placement to htbp
\makeatletter
\def\fps@figure{htbp}
\makeatother
\setlength{\emergencystretch}{3em} % prevent overfull lines
\providecommand{\tightlist}{%
  \setlength{\itemsep}{0pt}\setlength{\parskip}{0pt}}
\setcounter{secnumdepth}{-\maxdimen} % remove section numbering
\ifluatex
  \usepackage{selnolig}  % disable illegal ligatures
\fi

\title{HW1}
\author{}
\date{\vspace{-2.5em}}

\begin{document}
\maketitle

\begin{Shaded}
\begin{Highlighting}[]
\NormalTok{knitr}\SpecialCharTok{::}\NormalTok{opts\_chunk}\SpecialCharTok{$}\FunctionTok{set}\NormalTok{(}\AttributeTok{echo =} \ConstantTok{TRUE}\NormalTok{)}
\FunctionTok{library}\NormalTok{(fds)}
\end{Highlighting}
\end{Shaded}

\begin{verbatim}
## Warning: 패키지 'fds'는 R 버전 4.1.3에서 작성되었습니다
\end{verbatim}

\begin{verbatim}
## 필요한 패키지를 로딩중입니다: rainbow
\end{verbatim}

\begin{verbatim}
## Warning: 패키지 'rainbow'는 R 버전 4.1.3에서 작성되었습니다
\end{verbatim}

\begin{verbatim}
## 필요한 패키지를 로딩중입니다: MASS
\end{verbatim}

\begin{verbatim}
## 필요한 패키지를 로딩중입니다: pcaPP
\end{verbatim}

\begin{verbatim}
## Warning: 패키지 'pcaPP'는 R 버전 4.1.3에서 작성되었습니다
\end{verbatim}

\begin{verbatim}
## 필요한 패키지를 로딩중입니다: RCurl
\end{verbatim}

\begin{verbatim}
## Warning: 패키지 'RCurl'는 R 버전 4.1.3에서 작성되었습니다
\end{verbatim}

\begin{Shaded}
\begin{Highlighting}[]
\FunctionTok{library}\NormalTok{(fda)}
\end{Highlighting}
\end{Shaded}

\begin{verbatim}
## Warning: 패키지 'fda'는 R 버전 4.1.3에서 작성되었습니다
\end{verbatim}

\begin{verbatim}
## 필요한 패키지를 로딩중입니다: splines
\end{verbatim}

\begin{verbatim}
## 필요한 패키지를 로딩중입니다: deSolve
\end{verbatim}

\begin{verbatim}
## Warning: 패키지 'deSolve'는 R 버전 4.1.3에서 작성되었습니다
\end{verbatim}

\begin{verbatim}
## 
## 다음의 패키지를 부착합니다: 'fda'
\end{verbatim}

\begin{verbatim}
## The following object is masked from 'package:graphics':
## 
##     matplot
\end{verbatim}

\begin{Shaded}
\begin{Highlighting}[]
\FunctionTok{library}\NormalTok{(ggplot2)}
\end{Highlighting}
\end{Shaded}

\begin{verbatim}
## Warning: 패키지 'ggplot2'는 R 버전 4.1.2에서 작성되었습니다
\end{verbatim}

\begin{Shaded}
\begin{Highlighting}[]
\FunctionTok{library}\NormalTok{(fields)}
\end{Highlighting}
\end{Shaded}

\begin{verbatim}
## Warning: 패키지 'fields'는 R 버전 4.1.3에서 작성되었습니다
\end{verbatim}

\begin{verbatim}
## 필요한 패키지를 로딩중입니다: spam
\end{verbatim}

\begin{verbatim}
## Warning: 패키지 'spam'는 R 버전 4.1.3에서 작성되었습니다
\end{verbatim}

\begin{verbatim}
## Spam version 2.9-1 (2022-08-07) is loaded.
## Type 'help( Spam)' or 'demo( spam)' for a short introduction 
## and overview of this package.
## Help for individual functions is also obtained by adding the
## suffix '.spam' to the function name, e.g. 'help( chol.spam)'.
\end{verbatim}

\begin{verbatim}
## 
## 다음의 패키지를 부착합니다: 'spam'
\end{verbatim}

\begin{verbatim}
## The following objects are masked from 'package:base':
## 
##     backsolve, forwardsolve
\end{verbatim}

\begin{verbatim}
## 필요한 패키지를 로딩중입니다: viridis
\end{verbatim}

\begin{verbatim}
## Warning: 패키지 'viridis'는 R 버전 4.1.2에서 작성되었습니다
\end{verbatim}

\begin{verbatim}
## 필요한 패키지를 로딩중입니다: viridisLite
\end{verbatim}

\begin{verbatim}
## Warning: 패키지 'viridisLite'는 R 버전 4.1.2에서 작성되었습니다
\end{verbatim}

\begin{verbatim}
## 
## Try help(fields) to get started.
\end{verbatim}

\begin{Shaded}
\begin{Highlighting}[]
\FunctionTok{library}\NormalTok{(expm)}
\end{Highlighting}
\end{Shaded}

\begin{verbatim}
## Warning: 패키지 'expm'는 R 버전 4.1.3에서 작성되었습니다
\end{verbatim}

\begin{verbatim}
## 필요한 패키지를 로딩중입니다: Matrix
\end{verbatim}

\begin{verbatim}
## 
## 다음의 패키지를 부착합니다: 'Matrix'
\end{verbatim}

\begin{verbatim}
## The following object is masked from 'package:spam':
## 
##     det
\end{verbatim}

\begin{verbatim}
## 
## 다음의 패키지를 부착합니다: 'expm'
\end{verbatim}

\begin{verbatim}
## The following object is masked from 'package:Matrix':
## 
##     expm
\end{verbatim}

\hypertarget{chapter-1}{%
\section{Chapter 1}\label{chapter-1}}

\hypertarget{problem-1}{%
\subsection{Problem 1}\label{problem-1}}

\hypertarget{a-convert-the-pinch-data-to-functional-objects-using-15-b-splines-of-order-four-cubic-splines-and-plot-the-20-smoothed-curves-on-one-graph.}{%
\subsubsection{a) Convert the pinch data to functional objects using 15
B-splines of order four (cubic splines) and plot the 20 smoothed curves
on one
graph.}\label{a-convert-the-pinch-data-to-functional-objects-using-15-b-splines-of-order-four-cubic-splines-and-plot-the-20-smoothed-curves-on-one-graph.}}

\begin{Shaded}
\begin{Highlighting}[]
\NormalTok{time }\OtherTok{=}\NormalTok{ pinchtime}
\NormalTok{basis }\OtherTok{\textless{}{-}} \FunctionTok{create.bspline.basis}\NormalTok{(}\FunctionTok{c}\NormalTok{(}\DecValTok{0}\NormalTok{,}\FloatTok{0.3}\NormalTok{),}\AttributeTok{nbasis=}\DecValTok{15}\NormalTok{, }\AttributeTok{norder =} \DecValTok{4}\NormalTok{)}
\NormalTok{pinch.F}\OtherTok{\textless{}{-}}\FunctionTok{Data2fd}\NormalTok{(time, pinch, basis)}
\FunctionTok{plot}\NormalTok{(pinch.F)}
\end{Highlighting}
\end{Shaded}

\includegraphics{HW1_files/figure-latex/unnamed-chunk-1-1.pdf}

\begin{verbatim}
## [1] "done"
\end{verbatim}

\hypertarget{b-calculate-the-pointwise-mean-and-sd-and-add-them-to-the-plot.}{%
\subsubsection{b) Calculate the pointwise mean and SD and add them to
the
plot.}\label{b-calculate-the-pointwise-mean-and-sd-and-add-them-to-the-plot.}}

\begin{Shaded}
\begin{Highlighting}[]
\FunctionTok{mean.fd}\NormalTok{(pinch.F)}
\end{Highlighting}
\end{Shaded}

\begin{verbatim}
## $coefs
##              mean
##  [1,] -0.19640318
##  [2,]  0.53667546
##  [3,] -1.49889015
##  [4,]  2.68956848
##  [5,] 10.92286613
##  [6,]  6.04721931
##  [7,]  2.28283849
##  [8,]  0.35324569
##  [9,] -0.09900356
## [10,] -0.16429367
## [11,] -0.11493253
## [12,] -0.15313847
## [13,] -0.10891391
## [14,] -0.15580597
## [15,] -0.12331208
## 
## $basis
## $call
## basisfd(type = type, rangeval = rangeval, nbasis = nbasis, params = params, 
##     dropind = dropind, quadvals = quadvals, values = values, 
##     basisvalues = basisvalues)
## 
## $type
## [1] "bspline"
## 
## $rangeval
## [1] 0.0 0.3
## 
## $nbasis
## [1] 15
## 
## $params
##  [1] 0.025 0.050 0.075 0.100 0.125 0.150 0.175 0.200 0.225 0.250 0.275
## 
## $dropind
## NULL
## 
## $quadvals
## NULL
## 
## $values
## list()
## 
## $basisvalues
## list()
## 
## $names
##  [1] "bspl4.1"  "bspl4.2"  "bspl4.3"  "bspl4.4"  "bspl4.5"  "bspl4.6" 
##  [7] "bspl4.7"  "bspl4.8"  "bspl4.9"  "bspl4.10" "bspl4.11" "bspl4.12"
## [13] "bspl4.13" "bspl4.14" "bspl4.15"
## 
## attr(,"class")
## [1] "basisfd"
## 
## $fdnames
## $fdnames$time
##   [1]   1   2   3   4   5   6   7   8   9  10  11  12  13  14  15  16  17  18
##  [19]  19  20  21  22  23  24  25  26  27  28  29  30  31  32  33  34  35  36
##  [37]  37  38  39  40  41  42  43  44  45  46  47  48  49  50  51  52  53  54
##  [55]  55  56  57  58  59  60  61  62  63  64  65  66  67  68  69  70  71  72
##  [73]  73  74  75  76  77  78  79  80  81  82  83  84  85  86  87  88  89  90
##  [91]  91  92  93  94  95  96  97  98  99 100 101 102 103 104 105 106 107 108
## [109] 109 110 111 112 113 114 115 116 117 118 119 120 121 122 123 124 125 126
## [127] 127 128 129 130 131 132 133 134 135 136 137 138 139 140 141 142 143 144
## [145] 145 146 147 148 149 150 151
## 
## $fdnames$reps
## [1] "mean"
## 
## $fdnames$values
## [1] "mean value"
## 
## 
## attr(,"class")
## [1] "fd"
\end{verbatim}

\begin{Shaded}
\begin{Highlighting}[]
\FunctionTok{std.fd}\NormalTok{(pinch.F)}
\end{Highlighting}
\end{Shaded}

\begin{verbatim}
## $coefs
##              [,1]
##  [1,]  0.07975281
##  [2,]  0.09865960
##  [3,] -0.08021130
##  [4,]  0.95808616
##  [5,]  0.64742807
##  [6,]  1.11728335
##  [7,]  0.49878254
##  [8,]  0.32483098
##  [9,]  0.11764947
## [10,]  0.12235584
## [11,]  0.06434126
## [12,]  0.08597867
## [13,]  0.08325722
## [14,]  0.07574510
## [15,]  0.07045293
## 
## $basis
## $call
## basisfd(type = type, rangeval = rangeval, nbasis = nbasis, params = params, 
##     dropind = dropind, quadvals = quadvals, values = values, 
##     basisvalues = basisvalues)
## 
## $type
## [1] "bspline"
## 
## $rangeval
## [1] 0.0 0.3
## 
## $nbasis
## [1] 15
## 
## $params
##  [1] 0.025 0.050 0.075 0.100 0.125 0.150 0.175 0.200 0.225 0.250 0.275
## 
## $dropind
## NULL
## 
## $quadvals
## NULL
## 
## $values
## list()
## 
## $basisvalues
## list()
## 
## $names
##  [1] "bspl4.1"  "bspl4.2"  "bspl4.3"  "bspl4.4"  "bspl4.5"  "bspl4.6" 
##  [7] "bspl4.7"  "bspl4.8"  "bspl4.9"  "bspl4.10" "bspl4.11" "bspl4.12"
## [13] "bspl4.13" "bspl4.14" "bspl4.15"
## 
## attr(,"class")
## [1] "basisfd"
## 
## $fdnames
## $fdnames$time
##   [1]   1   2   3   4   5   6   7   8   9  10  11  12  13  14  15  16  17  18
##  [19]  19  20  21  22  23  24  25  26  27  28  29  30  31  32  33  34  35  36
##  [37]  37  38  39  40  41  42  43  44  45  46  47  48  49  50  51  52  53  54
##  [55]  55  56  57  58  59  60  61  62  63  64  65  66  67  68  69  70  71  72
##  [73]  73  74  75  76  77  78  79  80  81  82  83  84  85  86  87  88  89  90
##  [91]  91  92  93  94  95  96  97  98  99 100 101 102 103 104 105 106 107 108
## [109] 109 110 111 112 113 114 115 116 117 118 119 120 121 122 123 124 125 126
## [127] 127 128 129 130 131 132 133 134 135 136 137 138 139 140 141 142 143 144
## [145] 145 146 147 148 149 150 151
## 
## $fdnames$`Std. Dev.`
##  [1] "rep1"  "rep2"  "rep3"  "rep4"  "rep5"  "rep6"  "rep7"  "rep8"  "rep9" 
## [10] "rep10" "rep11" "rep12" "rep13" "rep14" "rep15" "rep16" "rep17" "rep18"
## [19] "rep19" "rep20"
## 
## $fdnames$`Std. Dev. values`
## [1] "value"
## 
## 
## attr(,"class")
## [1] "fd"
\end{verbatim}

\begin{Shaded}
\begin{Highlighting}[]
\FunctionTok{plot}\NormalTok{(pinch.F, }\AttributeTok{col =} \StringTok{"grey"}\NormalTok{)}
\end{Highlighting}
\end{Shaded}

\begin{verbatim}
## [1] "done"
\end{verbatim}

\begin{Shaded}
\begin{Highlighting}[]
\FunctionTok{plot}\NormalTok{(}\FunctionTok{mean.fd}\NormalTok{(pinch.F), }\AttributeTok{lwd =}\DecValTok{2}\NormalTok{, }\AttributeTok{add =} \ConstantTok{TRUE}\NormalTok{)}
\end{Highlighting}
\end{Shaded}

\begin{verbatim}
## [1] "done"
\end{verbatim}

\begin{Shaded}
\begin{Highlighting}[]
\FunctionTok{plot}\NormalTok{(}\FunctionTok{std.fd}\NormalTok{(pinch.F), }\AttributeTok{lwd =}\DecValTok{2}\NormalTok{, }\AttributeTok{add =} \ConstantTok{TRUE}\NormalTok{, }\AttributeTok{col =} \StringTok{"red"}\NormalTok{)}
\end{Highlighting}
\end{Shaded}

\includegraphics{HW1_files/figure-latex/unnamed-chunk-2-1.pdf}

\begin{verbatim}
## [1] "done"
\end{verbatim}

\hypertarget{c-graph-the-perspective-and-contour-plots-of-the-sample-covariance-function-ux2c6ct-s-of-the-pinch-curves.}{%
\subsubsection{c) Graph the perspective and contour plots of the sample
covariance function ˆc(t, s) of the pinch
curves.}\label{c-graph-the-perspective-and-contour-plots-of-the-sample-covariance-function-ux2c6ct-s-of-the-pinch-curves.}}

\begin{Shaded}
\begin{Highlighting}[]
\NormalTok{pinch\_var}\OtherTok{\textless{}{-}}\FunctionTok{var.fd}\NormalTok{(pinch.F)}
\NormalTok{pts}\OtherTok{\textless{}{-}}\FunctionTok{seq}\NormalTok{(}\AttributeTok{from=}\DecValTok{0}\NormalTok{, }\AttributeTok{to=}\FloatTok{0.3}\NormalTok{, }\AttributeTok{length =} \DecValTok{50}\NormalTok{)}
\NormalTok{pinch\_mat }\OtherTok{=} \FunctionTok{eval.bifd}\NormalTok{(pts, pts, pinch\_var)}
\FunctionTok{persp}\NormalTok{(pts, pts, pinch\_mat)}
\end{Highlighting}
\end{Shaded}

\includegraphics{HW1_files/figure-latex/unnamed-chunk-3-1.pdf}

\begin{Shaded}
\begin{Highlighting}[]
\FunctionTok{contour}\NormalTok{(pts, pts, pinch\_mat)}
\end{Highlighting}
\end{Shaded}

\includegraphics{HW1_files/figure-latex/unnamed-chunk-3-2.pdf}

\hypertarget{d-graph-the-first-four-efpcs-of-the-pinch-data.-how-many-components-do-you-need-to-explain-90-of-variation}{%
\subsubsection{d) Graph the first four EFPC's of the pinch data. How
many components do you need to explain 90\% of
variation?}\label{d-graph-the-first-four-efpcs-of-the-pinch-data.-how-many-components-do-you-need-to-explain-90-of-variation}}

\begin{Shaded}
\begin{Highlighting}[]
\NormalTok{pinch.pca }\OtherTok{=} \FunctionTok{pca.fd}\NormalTok{(pinch.F, }\AttributeTok{nharm=}\DecValTok{4}\NormalTok{)}
\NormalTok{pinch.pca}\SpecialCharTok{$}\NormalTok{varprop}
\end{Highlighting}
\end{Shaded}

\begin{verbatim}
## [1] 0.67408784 0.24791744 0.04585583 0.01878534
\end{verbatim}

We need 2 components to explain 90\% of variation.

\hypertarget{problem-2}{%
\subsection{Problem 2}\label{problem-2}}

\hypertarget{a-on-one-graph-plot-the-interest-rates-xtj-for-january-1982-and-for-june-2009-against-the-maturity-terms-tj-.-how-do-the-interest-rates-in-these-two-months-compare}{%
\subsubsection{a) On one graph, plot the interest rates x(tj ) for
January 1982 and for June 2009 against the maturity terms tj . How do
the interest rates in these two months
compare?}\label{a-on-one-graph-plot-the-interest-rates-xtj-for-january-1982-and-for-june-2009-against-the-maturity-terms-tj-.-how-do-the-interest-rates-in-these-two-months-compare}}

\begin{Shaded}
\begin{Highlighting}[]
\NormalTok{yield }\OtherTok{=}\NormalTok{ FedYieldcurve}
\NormalTok{terms }\OtherTok{=}\NormalTok{ yield}\SpecialCharTok{$}\NormalTok{x}
\FunctionTok{plot}\NormalTok{(terms, yield}\SpecialCharTok{$}\NormalTok{y[,}\DecValTok{1}\NormalTok{], }\AttributeTok{pch=}\DecValTok{15}\NormalTok{, }\AttributeTok{ylab=}\StringTok{"Yield"}\NormalTok{, }\AttributeTok{ylim=}\FunctionTok{c}\NormalTok{(}\DecValTok{0}\NormalTok{,}\DecValTok{16}\NormalTok{))}
\FunctionTok{points}\NormalTok{(terms, yield}\SpecialCharTok{$}\NormalTok{y[,}\DecValTok{330}\NormalTok{], }\AttributeTok{pch=}\DecValTok{16}\NormalTok{)}
\end{Highlighting}
\end{Shaded}

\includegraphics{HW1_files/figure-latex/unnamed-chunk-5-1.pdf}

The interest rates for January 1982 is much bigger than the interest
rates for June 2009.

\hypertarget{b-convert-the-yield-data-to-functional-objects-using-bspline-basis-with-four-basis-functions.-calculate-and-plot-the-the-mean-yield-function.-what-is-the-average-behavior-of-interest-rates-as-a-function-of-the-maturity}{%
\subsubsection{b) Convert the yield data to functional objects using
bspline basis with four basis functions. Calculate and plot the the mean
yield function. What is the average behavior of interest rates as a
function of the
maturity?}\label{b-convert-the-yield-data-to-functional-objects-using-bspline-basis-with-four-basis-functions.-calculate-and-plot-the-the-mean-yield-function.-what-is-the-average-behavior-of-interest-rates-as-a-function-of-the-maturity}}

\begin{Shaded}
\begin{Highlighting}[]
\NormalTok{yield\_data }\OtherTok{=}\NormalTok{ yield}\SpecialCharTok{$}\NormalTok{y}
\NormalTok{basis }\OtherTok{\textless{}{-}} \FunctionTok{create.bspline.basis}\NormalTok{(}\FunctionTok{c}\NormalTok{(}\DecValTok{0}\NormalTok{,}\DecValTok{330}\NormalTok{), }\AttributeTok{nbasis=}\DecValTok{4}\NormalTok{)}
\NormalTok{yield.F }\OtherTok{\textless{}{-}} \FunctionTok{Data2fd}\NormalTok{(terms, yield\_data, basis)}
\FunctionTok{plot}\NormalTok{(yield.F)}
\end{Highlighting}
\end{Shaded}

\includegraphics{HW1_files/figure-latex/unnamed-chunk-6-1.pdf}

\begin{verbatim}
## [1] "done"
\end{verbatim}

\begin{Shaded}
\begin{Highlighting}[]
\FunctionTok{plot}\NormalTok{(yield.F, }\AttributeTok{col =} \StringTok{"grey"}\NormalTok{)}
\end{Highlighting}
\end{Shaded}

\begin{verbatim}
## [1] "done"
\end{verbatim}

\begin{Shaded}
\begin{Highlighting}[]
\FunctionTok{plot}\NormalTok{(}\FunctionTok{mean.fd}\NormalTok{(yield.F), }\AttributeTok{lwd =}\DecValTok{4}\NormalTok{, }\AttributeTok{add =} \ConstantTok{TRUE}\NormalTok{)}
\end{Highlighting}
\end{Shaded}

\includegraphics{HW1_files/figure-latex/unnamed-chunk-6-2.pdf}

\begin{verbatim}
## [1] "done"
\end{verbatim}

\begin{Shaded}
\begin{Highlighting}[]
\FunctionTok{mean.fd}\NormalTok{(yield.F)}
\end{Highlighting}
\end{Shaded}

\begin{verbatim}
## $coefs
##              mean
## [1,]  5.185258259
## [2,]  8.549438361
## [3,]  7.053882562
## [4,] -0.007704886
## 
## $basis
## $call
## basisfd(type = type, rangeval = rangeval, nbasis = nbasis, params = params, 
##     dropind = dropind, quadvals = quadvals, values = values, 
##     basisvalues = basisvalues)
## 
## $type
## [1] "bspline"
## 
## $rangeval
## [1]   0 330
## 
## $nbasis
## [1] 4
## 
## $params
## NULL
## 
## $dropind
## NULL
## 
## $quadvals
## NULL
## 
## $values
## list()
## 
## $basisvalues
## list()
## 
## $names
## [1] "bspl4.1" "bspl4.2" "bspl4.3" "bspl4.4"
## 
## attr(,"class")
## [1] "basisfd"
## 
## $fdnames
## $fdnames$time
## [1] "3"   "6"   "12"  "60"  "84"  "120"
## 
## $fdnames$reps
## [1] "mean"
## 
## $fdnames$values
## [1] "mean value"
## 
## 
## attr(,"class")
## [1] "fd"
\end{verbatim}

It increases as time increases and then decreases at a certain point.
This curve is well known as a yield curve.

\hypertarget{c-plot-the-first-principal-component-of-the-interest-rate-curves.-what-percentage-of-variance-does-this-component-explain-interpret-the-plot-and-the-percentage-of-variance.}{%
\subsubsection{c) Plot the first principal component of the interest
rate curves. What percentage of variance does this component explain?
Interpret the plot and the percentage of
variance.}\label{c-plot-the-first-principal-component-of-the-interest-rate-curves.-what-percentage-of-variance-does-this-component-explain-interpret-the-plot-and-the-percentage-of-variance.}}

\begin{Shaded}
\begin{Highlighting}[]
\NormalTok{yield.pca }\OtherTok{=} \FunctionTok{pca.fd}\NormalTok{(yield.F, }\AttributeTok{nharm=}\DecValTok{1}\NormalTok{)}
\FunctionTok{plot}\NormalTok{(yield.pca}\SpecialCharTok{$}\NormalTok{harmonics)}
\end{Highlighting}
\end{Shaded}

\includegraphics{HW1_files/figure-latex/unnamed-chunk-7-1.pdf}

\begin{verbatim}
## [1] "done"
\end{verbatim}

\begin{Shaded}
\begin{Highlighting}[]
\NormalTok{yield.pca}\SpecialCharTok{$}\NormalTok{varprop}
\end{Highlighting}
\end{Shaded}

\begin{verbatim}
## [1] 0.9999981
\end{verbatim}

The the component explains 99.99981\% of the varaince. It means that it
can be well explained by bspline basis.

\hypertarget{problem-6}{%
\subsection{Problem 6}\label{problem-6}}

Since

\hypertarget{chapter-2}{%
\section{Chapter 2}\label{chapter-2}}

\hypertarget{problem-1-1}{%
\subsection{Problem 1}\label{problem-1-1}}

Since

\hypertarget{problem-2-1}{%
\subsection{Problem 2}\label{problem-2-1}}

\hypertarget{a-smooth-the-interest-rates-yields-in-january-1982-using-a-bspline-basis-with-four-basis-functions.-plot-the-raw-and-smoothed-interest-rates-on-one-graph.}{%
\subsubsection{a) Smooth the interest rates (yields) in January 1982
using a B--spline basis with four basis functions. Plot the raw and
smoothed interest rates on one
graph.}\label{a-smooth-the-interest-rates-yields-in-january-1982-using-a-bspline-basis-with-four-basis-functions.-plot-the-raw-and-smoothed-interest-rates-on-one-graph.}}

\begin{Shaded}
\begin{Highlighting}[]
\NormalTok{yield }\OtherTok{=}\NormalTok{ FedYieldcurve}
\NormalTok{terms }\OtherTok{=}\NormalTok{ yield}\SpecialCharTok{$}\NormalTok{x}
\NormalTok{yield\_data }\OtherTok{=}\NormalTok{ yield}\SpecialCharTok{$}\NormalTok{y[,}\DecValTok{1}\NormalTok{]}
\NormalTok{basis }\OtherTok{\textless{}{-}} \FunctionTok{create.bspline.basis}\NormalTok{(}\FunctionTok{c}\NormalTok{(}\DecValTok{0}\NormalTok{,}\DecValTok{330}\NormalTok{), }\AttributeTok{nbasis=}\DecValTok{4}\NormalTok{)}
\NormalTok{yield\_smooth }\OtherTok{\textless{}{-}} \FunctionTok{smooth.basis}\NormalTok{(terms, yield\_data, basis)}
\FunctionTok{plot}\NormalTok{(terms, yield}\SpecialCharTok{$}\NormalTok{y[,}\DecValTok{1}\NormalTok{], }\AttributeTok{pch=}\DecValTok{15}\NormalTok{, }\AttributeTok{ylab=}\StringTok{"Yield"}\NormalTok{, }\AttributeTok{ylim=}\FunctionTok{c}\NormalTok{(}\DecValTok{0}\NormalTok{,}\DecValTok{16}\NormalTok{))}
\FunctionTok{plot}\NormalTok{(yield\_smooth, }\AttributeTok{add =} \ConstantTok{TRUE}\NormalTok{)}
\end{Highlighting}
\end{Shaded}

\includegraphics{HW1_files/figure-latex/unnamed-chunk-8-1.pdf}

\begin{verbatim}
## [1] "done"
\end{verbatim}

\hypertarget{b-refit-the-january-1982-yields-using-a-penalized-smoothing-based-on-six-basis-functions-as-many-as-data-points-with-with-the-smoothing-parameter-ux3bb-1-and-the-second-derivative-as-the-penalty-operator.-add-the-smooth-in-red-to-the-graph-you-obtained-in-part-a-and-comment-on-the-result.}{%
\subsubsection{b) Re--fit the January 1982 yields using a penalized
smoothing based on six basis functions (as many as data points) with
with the smoothing parameter λ = 1, and the second derivative as the
penalty operator. Add the smooth in red to the graph you obtained in
part (a) and comment on the
result.}\label{b-refit-the-january-1982-yields-using-a-penalized-smoothing-based-on-six-basis-functions-as-many-as-data-points-with-with-the-smoothing-parameter-ux3bb-1-and-the-second-derivative-as-the-penalty-operator.-add-the-smooth-in-red-to-the-graph-you-obtained-in-part-a-and-comment-on-the-result.}}

\begin{Shaded}
\begin{Highlighting}[]
\NormalTok{yield }\OtherTok{=}\NormalTok{ FedYieldcurve}
\NormalTok{terms }\OtherTok{=}\NormalTok{ yield}\SpecialCharTok{$}\NormalTok{x}
\NormalTok{yield\_data }\OtherTok{=}\NormalTok{ yield}\SpecialCharTok{$}\NormalTok{y[,}\DecValTok{1}\NormalTok{]}
\NormalTok{basis }\OtherTok{\textless{}{-}} \FunctionTok{create.bspline.basis}\NormalTok{(}\FunctionTok{c}\NormalTok{(}\DecValTok{0}\NormalTok{,}\DecValTok{330}\NormalTok{), }\AttributeTok{nbasis=}\DecValTok{4}\NormalTok{)}
\NormalTok{yield\_smooth }\OtherTok{\textless{}{-}} \FunctionTok{smooth.basis}\NormalTok{(terms, yield\_data, basis)}
\FunctionTok{plot}\NormalTok{(terms, yield}\SpecialCharTok{$}\NormalTok{y[,}\DecValTok{1}\NormalTok{], }\AttributeTok{pch=}\DecValTok{15}\NormalTok{, }\AttributeTok{ylab=}\StringTok{"Yield"}\NormalTok{, }\AttributeTok{ylim=}\FunctionTok{c}\NormalTok{(}\DecValTok{0}\NormalTok{,}\DecValTok{16}\NormalTok{))}
\FunctionTok{plot}\NormalTok{(yield\_smooth, }\AttributeTok{add =} \ConstantTok{TRUE}\NormalTok{)}
\end{Highlighting}
\end{Shaded}

\begin{verbatim}
## [1] "done"
\end{verbatim}

\begin{Shaded}
\begin{Highlighting}[]
\NormalTok{basis }\OtherTok{\textless{}{-}} \FunctionTok{create.bspline.basis}\NormalTok{(}\FunctionTok{c}\NormalTok{(}\DecValTok{0}\NormalTok{,}\DecValTok{330}\NormalTok{), }\AttributeTok{nbasis=}\DecValTok{6}\NormalTok{)}
\NormalTok{yield\_par }\OtherTok{\textless{}{-}} \FunctionTok{fdPar}\NormalTok{(basis, }\AttributeTok{Lfdobj =}\DecValTok{2}\NormalTok{, }\AttributeTok{lambda =} \DecValTok{1}\NormalTok{)}
\NormalTok{yield\_smooth }\OtherTok{\textless{}{-}} \FunctionTok{smooth.basis}\NormalTok{(terms, yield\_data, yield\_par)}
\FunctionTok{plot}\NormalTok{(yield\_smooth, }\AttributeTok{col =} \StringTok{"red"}\NormalTok{, }\AttributeTok{add =} \ConstantTok{TRUE}\NormalTok{)}
\end{Highlighting}
\end{Shaded}

\includegraphics{HW1_files/figure-latex/unnamed-chunk-9-1.pdf}

\begin{verbatim}
## [1] "done"
\end{verbatim}

It shows very similar output from a). It is because we are using bspline
method.

\hypertarget{c-repeat-part-b-with-several-other-smoothing-parameters-ux3bb.-which-ux3bb-gives-the-most-informative-smooth-curve}{%
\subsubsection{c) Repeat part (b) with several other smoothing
parameters λ. Which λ gives the most informative smooth
curve?}\label{c-repeat-part-b-with-several-other-smoothing-parameters-ux3bb.-which-ux3bb-gives-the-most-informative-smooth-curve}}

\begin{Shaded}
\begin{Highlighting}[]
\NormalTok{yield }\OtherTok{=}\NormalTok{ FedYieldcurve}
\NormalTok{terms }\OtherTok{=}\NormalTok{ yield}\SpecialCharTok{$}\NormalTok{x}
\NormalTok{yield\_data }\OtherTok{=}\NormalTok{ yield}\SpecialCharTok{$}\NormalTok{y[,}\DecValTok{1}\NormalTok{]}
\NormalTok{basis }\OtherTok{\textless{}{-}} \FunctionTok{create.bspline.basis}\NormalTok{(}\FunctionTok{c}\NormalTok{(}\DecValTok{0}\NormalTok{,}\DecValTok{330}\NormalTok{), }\AttributeTok{nbasis=}\DecValTok{4}\NormalTok{)}
\NormalTok{yield\_smooth }\OtherTok{\textless{}{-}} \FunctionTok{smooth.basis}\NormalTok{(terms, yield\_data, basis)}
\FunctionTok{plot}\NormalTok{(terms, yield}\SpecialCharTok{$}\NormalTok{y[,}\DecValTok{1}\NormalTok{], }\AttributeTok{pch=}\DecValTok{15}\NormalTok{, }\AttributeTok{ylab=}\StringTok{"Yield"}\NormalTok{, }\AttributeTok{ylim=}\FunctionTok{c}\NormalTok{(}\DecValTok{0}\NormalTok{,}\DecValTok{16}\NormalTok{))}
\FunctionTok{plot}\NormalTok{(yield\_smooth, }\AttributeTok{add =} \ConstantTok{TRUE}\NormalTok{)}
\end{Highlighting}
\end{Shaded}

\begin{verbatim}
## [1] "done"
\end{verbatim}

\begin{Shaded}
\begin{Highlighting}[]
\NormalTok{basis }\OtherTok{\textless{}{-}} \FunctionTok{create.bspline.basis}\NormalTok{(}\FunctionTok{c}\NormalTok{(}\DecValTok{0}\NormalTok{,}\DecValTok{330}\NormalTok{), }\AttributeTok{nbasis=}\DecValTok{6}\NormalTok{)}
\NormalTok{yield\_par }\OtherTok{\textless{}{-}} \FunctionTok{fdPar}\NormalTok{(basis, }\AttributeTok{Lfdobj =}\DecValTok{2}\NormalTok{, }\AttributeTok{lambda =} \FloatTok{0.1}\NormalTok{)}
\NormalTok{yield\_smooth }\OtherTok{\textless{}{-}} \FunctionTok{smooth.basis}\NormalTok{(terms, yield\_data, yield\_par)}
\FunctionTok{print}\NormalTok{(}\StringTok{"lambda = 0.1"}\NormalTok{)}
\end{Highlighting}
\end{Shaded}

\begin{verbatim}
## [1] "lambda = 0.1"
\end{verbatim}

\begin{Shaded}
\begin{Highlighting}[]
\NormalTok{yield\_smooth}\SpecialCharTok{$}\NormalTok{gcv}
\end{Highlighting}
\end{Shaded}

\begin{verbatim}
##      rep1 
## 0.6282152
\end{verbatim}

\begin{Shaded}
\begin{Highlighting}[]
\FunctionTok{plot}\NormalTok{(yield\_smooth, }\AttributeTok{col =} \StringTok{"red"}\NormalTok{, }\AttributeTok{add =} \ConstantTok{TRUE}\NormalTok{)}
\end{Highlighting}
\end{Shaded}

\begin{verbatim}
## [1] "done"
\end{verbatim}

\begin{Shaded}
\begin{Highlighting}[]
\NormalTok{basis }\OtherTok{\textless{}{-}} \FunctionTok{create.bspline.basis}\NormalTok{(}\FunctionTok{c}\NormalTok{(}\DecValTok{0}\NormalTok{,}\DecValTok{330}\NormalTok{), }\AttributeTok{nbasis=}\DecValTok{6}\NormalTok{)}
\NormalTok{yield\_par }\OtherTok{\textless{}{-}} \FunctionTok{fdPar}\NormalTok{(basis, }\AttributeTok{Lfdobj =}\DecValTok{2}\NormalTok{, }\AttributeTok{lambda =} \FloatTok{0.2}\NormalTok{)}
\NormalTok{yield\_smooth }\OtherTok{\textless{}{-}} \FunctionTok{smooth.basis}\NormalTok{(terms, yield\_data, yield\_par)}
\FunctionTok{print}\NormalTok{(}\StringTok{"lambda = 0.2"}\NormalTok{)}
\end{Highlighting}
\end{Shaded}

\begin{verbatim}
## [1] "lambda = 0.2"
\end{verbatim}

\begin{Shaded}
\begin{Highlighting}[]
\NormalTok{yield\_smooth}\SpecialCharTok{$}\NormalTok{gcv}
\end{Highlighting}
\end{Shaded}

\begin{verbatim}
##      rep1 
## 0.6283764
\end{verbatim}

\begin{Shaded}
\begin{Highlighting}[]
\FunctionTok{plot}\NormalTok{(yield\_smooth, }\AttributeTok{col =} \StringTok{"blue"}\NormalTok{, }\AttributeTok{add =} \ConstantTok{TRUE}\NormalTok{)}
\end{Highlighting}
\end{Shaded}

\begin{verbatim}
## [1] "done"
\end{verbatim}

\begin{Shaded}
\begin{Highlighting}[]
\NormalTok{basis }\OtherTok{\textless{}{-}} \FunctionTok{create.bspline.basis}\NormalTok{(}\FunctionTok{c}\NormalTok{(}\DecValTok{0}\NormalTok{,}\DecValTok{330}\NormalTok{), }\AttributeTok{nbasis=}\DecValTok{6}\NormalTok{)}
\NormalTok{yield\_par }\OtherTok{\textless{}{-}} \FunctionTok{fdPar}\NormalTok{(basis, }\AttributeTok{Lfdobj =}\DecValTok{2}\NormalTok{, }\AttributeTok{lambda =} \FloatTok{0.3}\NormalTok{)}
\NormalTok{yield\_smooth }\OtherTok{\textless{}{-}} \FunctionTok{smooth.basis}\NormalTok{(terms, yield\_data, yield\_par)}
\FunctionTok{print}\NormalTok{(}\StringTok{"lambda = 0.3"}\NormalTok{)}
\end{Highlighting}
\end{Shaded}

\begin{verbatim}
## [1] "lambda = 0.3"
\end{verbatim}

\begin{Shaded}
\begin{Highlighting}[]
\NormalTok{yield\_smooth}\SpecialCharTok{$}\NormalTok{gcv}
\end{Highlighting}
\end{Shaded}

\begin{verbatim}
##      rep1 
## 0.6283621
\end{verbatim}

\begin{Shaded}
\begin{Highlighting}[]
\FunctionTok{plot}\NormalTok{(yield\_smooth, }\AttributeTok{col =} \StringTok{"green"}\NormalTok{, }\AttributeTok{add =} \ConstantTok{TRUE}\NormalTok{)}
\end{Highlighting}
\end{Shaded}

\begin{verbatim}
## [1] "done"
\end{verbatim}

\begin{Shaded}
\begin{Highlighting}[]
\NormalTok{basis }\OtherTok{\textless{}{-}} \FunctionTok{create.bspline.basis}\NormalTok{(}\FunctionTok{c}\NormalTok{(}\DecValTok{0}\NormalTok{,}\DecValTok{330}\NormalTok{), }\AttributeTok{nbasis=}\DecValTok{6}\NormalTok{)}
\NormalTok{yield\_par }\OtherTok{\textless{}{-}} \FunctionTok{fdPar}\NormalTok{(basis, }\AttributeTok{Lfdobj =}\DecValTok{2}\NormalTok{, }\AttributeTok{lambda =} \FloatTok{0.5}\NormalTok{)}
\NormalTok{yield\_smooth }\OtherTok{\textless{}{-}} \FunctionTok{smooth.basis}\NormalTok{(terms, yield\_data, yield\_par)}
\FunctionTok{print}\NormalTok{(}\StringTok{"lambda = 0.5"}\NormalTok{)}
\end{Highlighting}
\end{Shaded}

\begin{verbatim}
## [1] "lambda = 0.5"
\end{verbatim}

\begin{Shaded}
\begin{Highlighting}[]
\NormalTok{yield\_smooth}\SpecialCharTok{$}\NormalTok{gcv}
\end{Highlighting}
\end{Shaded}

\begin{verbatim}
##      rep1 
## 0.6282281
\end{verbatim}

\begin{Shaded}
\begin{Highlighting}[]
\FunctionTok{plot}\NormalTok{(yield\_smooth, }\AttributeTok{col =} \StringTok{"purple"}\NormalTok{, }\AttributeTok{add =} \ConstantTok{TRUE}\NormalTok{)}
\end{Highlighting}
\end{Shaded}

\begin{verbatim}
## [1] "done"
\end{verbatim}

\begin{Shaded}
\begin{Highlighting}[]
\NormalTok{basis }\OtherTok{\textless{}{-}} \FunctionTok{create.bspline.basis}\NormalTok{(}\FunctionTok{c}\NormalTok{(}\DecValTok{0}\NormalTok{,}\DecValTok{330}\NormalTok{), }\AttributeTok{nbasis=}\DecValTok{6}\NormalTok{)}
\NormalTok{yield\_par }\OtherTok{\textless{}{-}} \FunctionTok{fdPar}\NormalTok{(basis, }\AttributeTok{Lfdobj =}\DecValTok{2}\NormalTok{, }\AttributeTok{lambda =} \DecValTok{1}\NormalTok{)}
\NormalTok{yield\_smooth }\OtherTok{\textless{}{-}} \FunctionTok{smooth.basis}\NormalTok{(terms, yield\_data, yield\_par)}
\FunctionTok{print}\NormalTok{(}\StringTok{"lambda = 1.0"}\NormalTok{)}
\end{Highlighting}
\end{Shaded}

\begin{verbatim}
## [1] "lambda = 1.0"
\end{verbatim}

\begin{Shaded}
\begin{Highlighting}[]
\NormalTok{yield\_smooth}\SpecialCharTok{$}\NormalTok{gcv}
\end{Highlighting}
\end{Shaded}

\begin{verbatim}
##      rep1 
## 0.6277712
\end{verbatim}

\begin{Shaded}
\begin{Highlighting}[]
\FunctionTok{plot}\NormalTok{(yield\_smooth, }\AttributeTok{col =} \StringTok{"orange"}\NormalTok{, }\AttributeTok{add =} \ConstantTok{TRUE}\NormalTok{)}
\end{Highlighting}
\end{Shaded}

\begin{verbatim}
## [1] "done"
\end{verbatim}

\begin{Shaded}
\begin{Highlighting}[]
\NormalTok{basis }\OtherTok{\textless{}{-}} \FunctionTok{create.bspline.basis}\NormalTok{(}\FunctionTok{c}\NormalTok{(}\DecValTok{0}\NormalTok{,}\DecValTok{330}\NormalTok{), }\AttributeTok{nbasis=}\DecValTok{6}\NormalTok{)}
\NormalTok{yield\_par }\OtherTok{\textless{}{-}} \FunctionTok{fdPar}\NormalTok{(basis, }\AttributeTok{Lfdobj =}\DecValTok{2}\NormalTok{, }\AttributeTok{lambda =} \DecValTok{2}\NormalTok{)}
\NormalTok{yield\_smooth }\OtherTok{\textless{}{-}} \FunctionTok{smooth.basis}\NormalTok{(terms, yield\_data, yield\_par)}
\FunctionTok{print}\NormalTok{(}\StringTok{"lambda = 2.0"}\NormalTok{)}
\end{Highlighting}
\end{Shaded}

\begin{verbatim}
## [1] "lambda = 2.0"
\end{verbatim}

\begin{Shaded}
\begin{Highlighting}[]
\NormalTok{yield\_smooth}\SpecialCharTok{$}\NormalTok{gcv}
\end{Highlighting}
\end{Shaded}

\begin{verbatim}
##     rep1 
## 0.626784
\end{verbatim}

\begin{Shaded}
\begin{Highlighting}[]
\FunctionTok{plot}\NormalTok{(yield\_smooth, }\AttributeTok{col =} \StringTok{"skyblue"}\NormalTok{, }\AttributeTok{add =} \ConstantTok{TRUE}\NormalTok{)}
\end{Highlighting}
\end{Shaded}

\includegraphics{HW1_files/figure-latex/unnamed-chunk-10-1.pdf}

\begin{verbatim}
## [1] "done"
\end{verbatim}

Comparing to the value of GCV, it seems like the case of
\(\lambda = 0.2\) gives us a best result in this case.

\hypertarget{problem-5}{%
\subsection{Problem 5}\label{problem-5}}

\hypertarget{a-simulate-a-functional-sample-over-the-unit-interval-each-with-a-sample-size-of-50-from-the-matern-process.-for-the-first-half-of-the-sample-set-the-mean-function-equal-to-the-the-bump-function-with-parameters-c0-r0-a0-38-14-5.-for-the-second-half-use-c0-r0-a0-58-14-5.-you-may-choose-the-values-for-the-matern-covariance-function-as-well-as-the-number-of-points-sampled-per-curve.-plot-all-of-the-curves-and-include-a-curve-for-the-overall-mean-function.}{%
\subsubsection{(a) Simulate a functional sample over the unit interval
each with a sample size of 50 from the Mat´ern process. For the first
half of the sample, set the mean function equal to the the bump function
with parameters (c0, r0, a0) = (3/8, 1/4, 5). For the second half use
(c0, r0, a0) = (5/8, 1/4, 5). You may choose the values for the Mat´ern
covariance function as well as the number of points sampled per curve.
Plot all of the curves and include a curve for the overall mean
function.}\label{a-simulate-a-functional-sample-over-the-unit-interval-each-with-a-sample-size-of-50-from-the-matern-process.-for-the-first-half-of-the-sample-set-the-mean-function-equal-to-the-the-bump-function-with-parameters-c0-r0-a0-38-14-5.-for-the-second-half-use-c0-r0-a0-58-14-5.-you-may-choose-the-values-for-the-matern-covariance-function-as-well-as-the-number-of-points-sampled-per-curve.-plot-all-of-the-curves-and-include-a-curve-for-the-overall-mean-function.}}

\hypertarget{b-align-the-curves-using-continuous-registration.-plot-the-resulting-curves-and-include-a-mean-function.-comment-on-any-differences-with-a-and-if-the-registered-curves-exhibit-any-odd-patterns.}{%
\subsubsection{b) Align the curves using continuous registration. Plot
the resulting curves and include a mean function. Comment on any
differences with (a) and if the registered curves exhibit any odd
patterns.}\label{b-align-the-curves-using-continuous-registration.-plot-the-resulting-curves-and-include-a-mean-function.-comment-on-any-differences-with-a-and-if-the-registered-curves-exhibit-any-odd-patterns.}}

\hypertarget{c-carry-out-an-fpca-with-one-pc-on-the-unaligned-and-aligned-curves-separately.-for-each-do-a-simple-linear-regression-of-the-score-onto-a-dummy-variable-coded-01-indicating-which-type-of-mean-the-function-had-i.e.-is-it-from-the-first-or-second-half-of-the-sample.-calculate-a-p-value-to-determine-if-the-estimated-slope-parameters-you-get-are-significant.-compare-with-the-aligned-and-unaligned-curves.-what-did-aligning-do-to-the-p-value-you-may-want-to-rerun-your-simulations-a-few-times-to-see-how-the-p-values-change.}{%
\subsubsection{c) Carry out an FPCA with one PC on the unaligned and
aligned curves separately. For each, do a simple linear regression of
the score onto a dummy variable (coded 0/1) indicating which type of
mean the function had (i.e.~is it from the first or second half of the
sample). Calculate a p-value to determine if the estimated slope
parameters you get are significant. Compare with the aligned and
unaligned curves. What did aligning do to the p-value? You may want to
rerun your simulations a few times to see how the p-values
change.}\label{c-carry-out-an-fpca-with-one-pc-on-the-unaligned-and-aligned-curves-separately.-for-each-do-a-simple-linear-regression-of-the-score-onto-a-dummy-variable-coded-01-indicating-which-type-of-mean-the-function-had-i.e.-is-it-from-the-first-or-second-half-of-the-sample.-calculate-a-p-value-to-determine-if-the-estimated-slope-parameters-you-get-are-significant.-compare-with-the-aligned-and-unaligned-curves.-what-did-aligning-do-to-the-p-value-you-may-want-to-rerun-your-simulations-a-few-times-to-see-how-the-p-values-change.}}

\hypertarget{d-come-up-with-one-potential-settingapplication-where-you-might-lose-something-if-you-align.-make-up-whatever-scenario-you-like-but-think-it-through.}{%
\subsubsection{d) Come up with one potential setting/application where
you might lose something if you align. Make up whatever scenario you
like, but think it
through.}\label{d-come-up-with-one-potential-settingapplication-where-you-might-lose-something-if-you-align.-make-up-whatever-scenario-you-like-but-think-it-through.}}

\end{document}
